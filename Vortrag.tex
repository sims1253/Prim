\documentclass[mathserif, compress]{beamer}
% Class options include: notes, handout, trans
%                        

% Theme for beamer presentation.
\usepackage{beamerthemelined} 
% Other themes include: beamerthemebars, beamerthemelined, 
%                       beamerthemetree, beamerthemeplain

\title{Primfaktorzerlegung und Primzahltests}    % Enter your title between curly braces
\author{Maximilian Scholz}                 % Enter your name between curly braces
\institute{Proseminar Mathematik}      % Enter your institute name between curly braces
\date{\today}                    % Enter the date or \today between curly braces

\setbeamertemplate{footline}[frame number]
\begin{document}

% Creates title page of slide show using above information
\begin{frame}
  \titlepage
\end{frame}

\section[]{}

% Creates table of contents slide incorporating
% all \section and \subsection commands
\begin{frame}
  \frametitle{Inhalt}
  \tableofcontents
\end{frame}

\section{Einleitung zu Primzahlen}
  
\begin{frame}
  \frametitle{Primzahlen}
  \begin{itemize}
    \item Nat\"urliche Zahlen $> 1$ die nur durch sich selbst und $1$ teilbar sind.
      \vspace{3mm}
    \item Es gibt unendlich viele Primzahlen. (Euklid)
      \vspace{3mm}
    \item Jede nat\"urliche Zahl l\"asst sich als Produkt von Primzalen darstellen. 
	  Bis auf die Reihenfolge ist diese Darstellung eindeutig. (Euklid)
  \end{itemize}
\end{frame}

\section{Sieb des Eratosthenes}

\begin{frame}
  \frametitle{Sieb des Eratosthenes}   % Insert frame title between curly braces

  \begin{itemize}
  \item<1-> Der kleinste Teiler $> 1$ einer zusammengesetzten Zahl $n$ ist eine Primzahl $p$. 
  \vspace{3mm}
  \item<2-> Da $p$ der kleinste Teiler ist, gilt $p \leq \frac{n}{p}$, also $ p^2 \leq n $.
  \vspace{3mm}
  \item<3-> Alle zusammengesetzten Zahlen $n<N$ werden also beim Sieben mit einer Siebzahl $q$ mit $q^2<n$ gestrichen.
  \vspace{3mm}
  \item<4-> Die \"ubrigen Zahlen sind also Primzahlen.
  \end{itemize}
\end{frame}

\begin{frame}
  \frametitle{Sieb des Eratosthenes}   % Insert frame title between curly braces
BILD
\end{frame}


\section{Pollard Rho Methode}

\subsection{Hase Igel Algorithmus}

\begin{frame}
  \frametitle{Hase Igel Algorithmus}
  \begin{itemize}
    \item<1-> Sei $M$ eine endliche Menge mit der Abbildung $f : M \rightarrow M$.
      \vspace{3mm}
    \item<2-> Man w\"ahle $x_0 \in M$ und erzeuge die Folge $x_0, x_1, x_2,...$ mit $x_{i+1} = f(x_i)$ .
      \vspace{3mm}
    \item<3-> $\exists \ i,j \in \mathbb{N}$, sodass $i \not= j$ und $x_i = x_j$ gilt.
      \vspace{3mm}
    \item<4-> Die Folge $y_0, y_1, y_2,...$ gegeben durch $y_0=x_0$ und $y_{i+1}=f(f(y_i))$ ist gleich der Folge $x_0,x_2,x_4,...$.
  \end{itemize}
\end{frame}

\begin{frame}
  \frametitle{Beweis Teil 1}
  \begin{itemize}
    \item<1->  $g:\mathbb{N} \rightarrow M \ \text{gegeben durch} \ g(n)=f^n(x_0)$
      \vspace{3mm}
    \item<2-> M ist beschr\"ankt also kann g nicht injektiv sein. Daraus folgt:\\
	      $\exists \ i,j \in \mathbb{N}, i\not=j \ \text{sodass} \ g(i)=g(j) \ \text{und damit} \ x_i=x_j$ bei $i\not=j$.
  \end{itemize}
\end{frame}

\begin{frame}
  \frametitle{Beweis Teil 2}
  \begin{itemize}
    \item<1-> Angenommen $x_i=x_j$ f\"ur $j>i$.
     \vspace{3mm}
    \item<2-> Falls $n\geq i$ und $2n=n+k(j-i)\geq i$ mit $k\geq 0$ muss $x_n=x_2n$ gelten.
     \vspace{3mm}
    \item<3-> Man w\"ahle $k\geq 0$ sodass $n=k(j-1)\geq i$ und erh\"alt das gesuchte $n$.
     \vspace{3mm}
    \item<4-> Aus $x_{m+2}=f(f(x_m))$ folgt $y_m = x_{2m}$.
  \end{itemize}
\end{frame}

\begin{frame}
  \frametitle{Konkruenz modulo p}
  \begin{itemize}
	\item<1->$a \equiv b \pmod p \Leftrightarrow p|(a-b)$
\vspace{3mm}
	\item<2->$a=p\cdot x +c,\ \ \ b= p\cdot y +c$
\vspace{3mm}
	\item<3->$a-b=p(x-y)+(c-c)=p(x-y)$
\vspace{3mm}
	\item<4->$p|p(x-y)$
  \end{itemize}
\end{frame}


\subsection{Pollard Rho Algorithmus}
\begin{frame}
  \frametitle{Pollard Rho Methode}
  \begin{itemize}
    \item<1-> Sei $N$ eine zusammengesetzte Zahl und \\$p$ ein Primfaktor von $N$.
      \vspace{3mm}
    \item<2-> Gesucht sind $0\leq a,b<N$ sodass $a \equiv b \pmod p$. \\Dann gilt $p|a-b$
      \vspace{3mm}
    \item<3-> Daraus folgt $1<ggT(a-b,N)\leq N$. \\Wenn $a\not = b$ gilt, ist $ggT(a-b,N)$ ein nichttrivialer Faktor von N.
  \end{itemize}
\end{frame}

\begin{frame}
  \frametitle{Pollard Rho Algorithmus}
  \begin{itemize}
    \item<1-> Sei $f(x)$ eine ganzzahlige Polynomfunktion und $S \in \mathbb{Z}$.
     \vspace{3mm}
    \item<2-> Man erzeuge eine Folge von Pseudozufallszahlen mit:\\ $x_0= S, \ x_{i+1}=f(x_i)\mod N$.
    \vspace{3mm}
    \item<3-> Wird schlie�lich periodisch, da beschr\"ankt.
    \vspace{3mm}
    \item<4-> Anstatt $x_k=^?y_k$ suchen wir nach $ggT(x_k - y_k)>^? 1$
  \end{itemize}
\end{frame}

\begin{frame}
  \frametitle{Pollard Rho Beispiel}
Gesucht: Primfaktorzerlegung von N=143\\
\ \\
Parameter: $x_0 = y_0 = 0, \ f(x)=(x^2+1)\mod N$
	$$
  \begin{array}{c|c|c|c}
	k & x_k = f(x_{k-1}) & y_k = f(f(y_{k-1})) & ggT(x_k - y_k, N) \\ \hline 
        0 & 0 & 0 & 0 \\ \hline
	1 & 1 & 2 & 1 \\ \hline
	2 & 2 & 26 & 1 \\ \hline
	3 & 5 & 15 & 1 \\ \hline
	4 & 26 & 26 & 0 \\ \hline
	5 & 105 & 15 & 1 \\ \hline
	6 & 15 & 26 & 11 \\ \hline
  \end{array}
	$$
\end{frame}

\subsection{Komplexit�t}
\begin{frame}
  \frametitle{Gebutstagsproblem}
  \begin{itemize}
    \item 
  \end{itemize}
\end{frame}

\begin{frame}
  \frametitle{Pollard Rho Komplexit�t}
  \begin{itemize}
    \item 
  \end{itemize}
\end{frame}

\end{document}
