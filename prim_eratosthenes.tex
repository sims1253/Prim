%this file is prim_eratosthenes.tex


	\section{Sieb des Eratosthenes}
	 \label{sec:eratosthenes}
	Das Sieb des Eratosthenes ist ein etwa 2000 Jahre altes Verfahren, um alle Primzahlen in einem gegebenen Zahlenbereich zu finden.
	 
	\subsection{Idee}
	Das Sieb des Eratosthenes bestimmt alle Primzahlen N, indem es alle zusammengesetzten Zahlen streicht. Daher kommt der Name: Sieb.\\
	Das Abbruchkriterium ist ein sch\"ones Beispiel, wie unn\"otiger Aufwand vermieden werden kann.
	
	\subsection{Beispiel}
	Bevor die genaue Funktionsweise behandelt wird, hier einmal ein Beispiel f\"ur die Funktionsweise der Methode.\\
	\begin{table}[!ht] 
		\centering
		
		\begin{tabular}{|c|c|c|c|c|c|c|c|c|c|}
			\hline
			0  &  1 &  2 &  3 &  4 &  5 &  6 &  7 &  8 &  9 \\
			\hline
			10 & 11 & 12 & 13 & 14 & 15 & 16 & 17 & 18 & 19 \\
			\hline
			20 & 21 & 22 & 23 & 24 & 25 &  &  &  &  \\
			\hline
		\end{tabular}
		\caption{Eratosthenes 1}
		\label{tab:eratosthenes1}
	\end{table}
	
	\noindent Zu Beginn werden alle Zahlen in dem Bereich, in dem nach Primzahlen gesucht werden soll aufgeschrieben. Hier $n=25$.
	
	\begin{table}[!ht] 
		\centering
		\begin{tabular}{|c|c|c|c|c|c|c|c|c|c|}
			\hline
			\cellcolor{red}0  &  \cellcolor{red}1 &  2 &  3 &  4 &  5 &  6 &  7 &  8 &  9 \\
			\hline
			10 & 11 & 12 & 13 & 14 & 15 & 16 & 17 & 18 & 19 \\
			\hline
			20 & 21 & 22 & 23 & 24 & 25 &  &  &  &  \\
			\hline
		\end{tabular}
		\caption{Eratosthenes 2}
		\label{tab:eratosthenes2}
	\end{table}
	
	\noindent Die Zahlen $0$ und $1$ sind per Definition keine Primzahlen, wir k\"onnen sie somit streichen.	 	
	
	\begin{table}[!ht] 
		\centering
		\begin{tabular}{|c|c|c|c|c|c|c|c|c|c|}
			\hline
			\cellcolor{red}0  & \cellcolor{red} 1 &  2 &  3 &  \cellcolor{red}4 &  5 & \cellcolor{red} 6 &  7 & \cellcolor{red} 8 &  9 \\
			\hline
			\cellcolor{red}10 & 11 & \cellcolor{red}12 & 13 & \cellcolor{red}14 & 15 & \cellcolor{red}16 & 17 & \cellcolor{red}18 & 19 \\
			\hline
			\cellcolor{red}20 & 21 & \cellcolor{red}22 & 23 & \cellcolor{red}24 & 25 & \ & \ & \ & \ \\
			\hline
		\end{tabular}
		\caption{Eratosthenes 3}
		\label{tab:eratosthenes3}
	\end{table}
	
	\noindent Die erste nicht gestrichene Zahl ist die $2$. Gleichzeitig ist die $2$ die erste Primzahl. Da alles Zahlen die durch $2$ teilbar sind nicht prim sind, k\"onnen wir alle geraden Zahlen streichen. 
	
	\begin{table}[!ht] 
		\centering
		\begin{tabular}{|c|c|c|c|c|c|c|c|c|c|}
			\hline	 		 			
			\cellcolor{red}0  & \cellcolor{red} 1 &  2 &  3 & \cellcolor{red}4 &  5 & \cellcolor{red} 6 &  7 & \cellcolor{red} 8 &  \cellcolor{red}9 \\
			\hline
			\cellcolor{red}10 & 11 & \cellcolor{red}12 & 13 & \cellcolor{red}14 & \cellcolor{red}15 & \cellcolor{red}16 & 17 & \cellcolor{red}18 & 19 \\
			\hline
			\cellcolor{red}20 & \cellcolor{red}21 & \cellcolor{red}22 & 23 & \cellcolor{red}24 & 25 &  &  &  &  \\
			\hline
		\end{tabular}
		\caption{Eratosthenes 4}
		\label{tab:eratosthenes4}
	\end{table}
	
	\noindent Die n\"achste nicht gestrichene Zahl ist die $3$. Somit ist auch die $3$ eine Primzahl. Wieder werden alle Vielfachen von $3$ gestrichen.
	
	\begin{table}[!ht] 
		\centering
		\begin{tabular}{|c|c|c|c|c|c|c|c|c|c|}
			\hline	 		 			
			\cellcolor{red}0  & \cellcolor{red} 1 &  2 &  3 & \cellcolor{red}4 &  5 & \cellcolor{red} 6 &  7 & \cellcolor{red} 8 &  \cellcolor{red}9 \\
			\hline
			\cellcolor{red}10 & 11 & \cellcolor{red}12 & 13 & \cellcolor{red}14 & \cellcolor{red}15 & \cellcolor{red}16 & 17 & \cellcolor{red}18 & 19 \\
			\hline
			\cellcolor{red}20 & \cellcolor{red}21 & \cellcolor{red}22 & 23 & \cellcolor{red}24 & \cellcolor{red}25 &  &  & &  \\
			\hline
		\end{tabular}
		\caption{Eratosthenes 5}
		\label{tab:eratosthenes5}
	\end{table}	
	
	\noindent Die $4$ ist schon gestrichen, da sie durch $2$ teilbar ist. Die n\"achste nicht gestrichene Zahl ist die $5$. Damit ist die $5$ die n\"achste Primzahl und alle Vielfachen von $5$ werden gestrichen.
	
	\begin{table}[!ht] 
		\centering
		\begin{tabular}{|c|c|c|c|c|c|c|c|c|c|}
			\hline	 		 			
			\cellcolor{red}0  & \cellcolor{red} 1 &  \cellcolor{green}2 &  \cellcolor{green}3 & \cellcolor{red}4 &  \cellcolor{green}5 & \cellcolor{red} 6 &  \cellcolor{green}7 & \cellcolor{red} 8 &  \cellcolor{red}9 \\
			\hline
			\cellcolor{red}10 & \cellcolor{green}11 & \cellcolor{red}12 & \cellcolor{green}13 & \cellcolor{red}14 & \cellcolor{red}15 & \cellcolor{red}16 & \cellcolor{green}17 & \cellcolor{red}18 & \cellcolor{green}19 \\
			\hline
			\cellcolor{red}20 & \cellcolor{red}21 & \cellcolor{red}22 & \cellcolor{green}23 & \cellcolor{red}24 & \cellcolor{red}25 &  &  &  &  \\
			\hline
		\end{tabular}
		\caption{Eratosthenes 6}
		\label{tab:eratosthenes6}			
	\end{table}
	
	\noindent Da nur die Zahlen bis $25$ betrachtet werden, ist die Methode an dieser Stelle fertig. Mit $5=\sqrt{25}$ haben wir das Abbruchkriterium erreicht und damit alle m\"oglichen Teiler der Zahlen $<25$ abgearbeitet. Die \"ubrigen Zahlen, die gr\"un unterlegt sind, sind Primzahlen. Wieso wir fertig sind, wenn wir alle Vielfachen von $5$ gestrichen haben, wird im n\"achsten Abschnitt deutlich.
	
	\subsection{Mathematik}
	\subsubsection{Funktionsweise}
	Die Methode des quadratischen Siebes l\"asst sich wie folgt beschreiben:
	\begin{enumerate}
		
		\item W\"ahle eine nat\"uriche Zahl $n > 1$. Dies ist die obere Grenze der Zahlen, in denen nach Primzahlen gesucht wird.
		\item Die kleinste noch nicht gestrichene Zahl $m$ mit $2 \leq m$ wird die neue Ausgangszahl. Im Beispiel sind das die $2$ bei in Tabelle 3, die $3$ bei Tabelle 4 und die $5$ bei Tabelle 5.
		\item Solange $m^2 \leq n$ gilt, streiche alle Vielfachen $c\cdot m$ ($c \in \mathbb{N} $ ) mit $m^2 \leq cm \leq n$. Im Beispiel durch die rote F\"arbung zu erkennen.
		\item \"Ubrig bleiben alle Primzahlen zwischen $0$ und $n$. Im Beispiel durch die gr\"une F\"arbung zu erkennen.
	\end{enumerate}
	
	\noindent Interessant ist zum einen das Abbruchkriterium. Wieso reicht es $m\leq \sqrt{n}$ zu betrachten um alle Primzahlen bis $n$ zu finden. Au\ss erdem sollte die intuitive Funktionsweise des Streichens von Vielfachen nicht als Beweis daf\"ur gesehen werden, dass die Methode des Siebens wirklich in jedem Fall funktioniert. F\"ur diese beiden Teile wird im folgenden ein Beweise gezeigt.

	\subsubsection{Beweis}
	Beide Teile lassen sich zusammen beweisen:\\
	Der kleinste Teiler einer zusammengesetzten Zahl $n$ ist eine Primzahl $p$. Es gilt $p>1$.\\
	Da $p | n$ gilt, gilt auch $\frac{n}{p}|n$. Das Ergebnis von $\frac{n}{p}$ ist nat\"urlich auch ein Teiler von $n$, da es mit $p$ multipliziert $n$ ergibt.\\
	Da $p$ der kleinste Teiler ist, gilt $p \leq \frac{n}{p}$, da jeder andere Teiler mindestens so gro\ss wie $p$ sein muss. Daraus folgt: 
	\[p^2 \leq n\]	
	Wenn wir mit dem Sieb des Eratosthenes nach Primzahlen von $1$ bis $N$ suchen, werden also
	alle zusammengesetzten Zahlen $n < N$  beim Sieben mit einer Zahl $p$, f\"ur die gilt $p \leq \sqrt{n}$, gestrichen.\\
	Die \"ubrigen Zahlen m\"ussen also Primzahlen sein.
