%this file is prim_eratosthenes.tex


	\section{Sieb des Eratosthenes}
	 \label{sec:eratosthenes}
	Das Sieb des Eratosthenes ist ein etwa 2000 Jahre altes Verfahren, um alle Primzahlen in einem gegebenen Zahlenbereich zu finden. 
	\subsection{Funktionsweise}
	Das Sieb des Eratosthenes bestimmt alle Primzahlen N, indem es alle zusammengesetzten Zahlen streicht, daher der Name: Sieb.\\
	Das Abbruchkriterium ist ein sch\"önes Beispiel wie unn\"ötiger Aufwand vermieden werden kann.
	\subsection{Beispiel}
	Bevor die genaue Funktionsweise behandelt wird, hier einmal ein Beispiel f\"ür die Funktionsweise der Methode.\\
	\ \\
	\ \\
	\begin{table}[!ht] 
		\centering
		
		\begin{tabular}{|c|c|c|c|c|c|c|c|c|c|}
			\hline
			0  &  1 &  2 &  3 &  4 &  5 &  6 &  7 &  8 &  9 \\
			\hline
			10 & 11 & 12 & 13 & 14 & 15 & 16 & 17 & 18 & 19 \\
			\hline
			20 & 21 & 22 & 23 & 24 & 25 &  &  &  &  \\
			\hline
		\end{tabular}
		\caption{Eratosthenes 1}
		\label{tab:eratosthenes1}
	\end{table}
	
	Zu Beginn werden alle Zahlen im Bereich, in dem nach Primzahlen gesucht werden soll aufgeschrieben.
	
	\begin{table}[!ht] 
		\centering
		\begin{tabular}{|c|c|c|c|c|c|c|c|c|c|}
			\hline
			\cellcolor{red}0  &  \cellcolor{red}1 &  2 &  3 &  4 &  5 &  6 &  7 &  8 &  9 \\
			\hline
			10 & 11 & 12 & 13 & 14 & 15 & 16 & 17 & 18 & 19 \\
			\hline
			20 & 21 & 22 & 23 & 24 & 25 &  &  &  &  \\
			\hline
		\end{tabular}
		\caption{Eratosthenes 2}
		\label{tab:eratosthenes2}
	\end{table}
	
	Die Zahlen $0$ und $1$ sind per Definition keine Primzahlen, wir können sie somit streichen.	 	
	
	\begin{table}[!ht] 
		\centering
		\begin{tabular}{|c|c|c|c|c|c|c|c|c|c|}
			\hline
			\cellcolor{red}0  & \cellcolor{red} 1 &  2 &  3 &  \cellcolor{red}4 &  5 & \cellcolor{red} 6 &  7 & \cellcolor{red} 8 &  9 \\
			\hline
			\cellcolor{red}10 & 11 & \cellcolor{red}12 & 13 & \cellcolor{red}14 & 15 & \cellcolor{red}16 & 17 & \cellcolor{red}18 & 19 \\
			\hline
			\cellcolor{red}20 & 21 & \cellcolor{red}22 & 23 & \cellcolor{red}24 & 25 & \ & \ & \ & \ \\
			\hline
		\end{tabular}
		\caption{Eratosthenes 3}
		\label{tab:eratosthenes3}
	\end{table}
	
	Die erste nicht gestrichene Zahl ist $2$. Gleichzeitig ist $2$ die erste Primzahl. Da alles Zahlen die durch $2$ teilbar sind nicht prim sind, können wir alle geraden Zahlen streichen. 
	
	\begin{table}[!ht] 
		\centering
		\begin{tabular}{|c|c|c|c|c|c|c|c|c|c|}
			\hline	 		 			
			\cellcolor{red}0  & \cellcolor{red} 1 &  2 &  3 & \cellcolor{red}4 &  5 & \cellcolor{red} 6 &  7 & \cellcolor{red} 8 &  \cellcolor{red}9 \\
			\hline
			\cellcolor{red}10 & 11 & \cellcolor{red}12 & 13 & \cellcolor{red}14 & \cellcolor{red}15 & \cellcolor{red}16 & 17 & \cellcolor{red}18 & 19 \\
			\hline
			\cellcolor{red}20 & \cellcolor{red}21 & \cellcolor{red}22 & 23 & \cellcolor{red}24 & 25 &  &  &  &  \\
			\hline
		\end{tabular}
		\caption{Eratosthenes 4}
		\label{tab:eratosthenes4}
	\end{table}
	
	Die n\"achste nicht gestrichene Zahl ist $3$. Somit ist auch $3$ eine Primzahl. Wieder werden alle Vielfachen von $3$ gestrichen, da sie durch $3$ teilbar, und somit nicht prim, sind.
	
	\begin{table}[!ht] 
		\centering
		\begin{tabular}{|c|c|c|c|c|c|c|c|c|c|}
			\hline	 		 			
			\cellcolor{red}0  & \cellcolor{red} 1 &  2 &  3 & \cellcolor{red}4 &  5 & \cellcolor{red} 6 &  7 & \cellcolor{red} 8 &  \cellcolor{red}9 \\
			\hline
			\cellcolor{red}10 & 11 & \cellcolor{red}12 & 13 & \cellcolor{red}14 & \cellcolor{red}15 & \cellcolor{red}16 & 17 & \cellcolor{red}18 & 19 \\
			\hline
			\cellcolor{red}20 & \cellcolor{red}21 & \cellcolor{red}22 & 23 & \cellcolor{red}24 & \cellcolor{red}25 &  &  & &  \\
			\hline
		\end{tabular}
		\caption{Eratosthenes 5}
		\label{tab:eratosthenes5}
	\end{table}	
	
	Die $4$ ist schon gestrichen, da sie durch $2$ teilbar ist. Die n\"achste nicht gestrichene Zahl ist $5$. Damit ist $5$ die n\"achste Primzahl und alle Vielfachen von $5$ werden gestrichen.
	
	\begin{table}[!ht] 
		\centering
		\begin{tabular}{|c|c|c|c|c|c|c|c|c|c|}
			\hline	 		 			
			\cellcolor{red}0  & \cellcolor{red} 1 &  \cellcolor{green}2 &  \cellcolor{green}3 & \cellcolor{red}4 &  \cellcolor{green}5 & \cellcolor{red} 6 &  \cellcolor{green}7 & \cellcolor{red} 8 &  \cellcolor{red}9 \\
			\hline
			\cellcolor{red}10 & \cellcolor{green}11 & \cellcolor{red}12 & \cellcolor{green}13 & \cellcolor{red}14 & \cellcolor{red}15 & \cellcolor{red}16 & \cellcolor{green}17 & \cellcolor{red}18 & \cellcolor{green}19 \\
			\hline
			\cellcolor{red}20 & \cellcolor{red}21 & \cellcolor{red}22 & \cellcolor{green}23 & \cellcolor{red}24 & \cellcolor{red}25 &  &  &  &  \\
			\hline
		\end{tabular}
		\caption{Eratosthenes 6}
		\label{tab:eratosthenes6}			
	\end{table}
	
	Da nur die Zahlen bis $25$ betrachtet werden, ist die Methode an dieser Stelle fertig. Mit $5=\sqrt{25}$ haben wir den kleinsten Teiler von 25 gefunden und damit alle m\"oglichen Teiler der Zahlen $<25$. 
	
	\subsection{Mathematik}
	Die Methode des quadratischen Siebes l\"asst sich wie folgt beschreiben:
	\begin{itemize}
		
		\item W\"ahle eine nat\"uriche Zahl $n > 1$. Dies ist die obere Grenze der Zahlen, in denen nach Primzahlen gesucht wird.
		\item Die kleinste noch nicht gestrichene Zahl $m$ mit $2 \leq m$ wird die aktuelle Zahl.
		\item Wenn $m^2 \leq n$ ist, streiche alle Vielfachen $c\cdot m$ ($c \in \mathbb{N} $ ) mit $m^2 \leq cm \leq n$
		\item \"Ubrig bleiben alle Primzahlen zwischen $0$ und $n$. 
	\end{itemize}