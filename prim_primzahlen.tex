 % this file is prim_primzahlen.tex	
	
	\section{Einf\"uhrung in Primzahlen} 
	\label{sec:primzahlen}
	Die wichtigsten Dinge, die es zu Primzahlen im Bezug auf diesen Text gibt lassen sich wie folgt zusammenfassen:

	\begin{itemize}
		\item Definition: Jede nat\"urliche Zahl $>1$, die nur von sich selber und  $1$ geteilt wird, ist eine Primzahl. Dies l\"asst sich auch so beschreiben, dass eine Primzahl nur beim Teilen durch $1$ und sich selbst keinen Rest hat.
		
		\item Der griechische Mathematiker Euklid hat bewiesen, dass es unendlich viele Primzahlen gibt. W\"urde dies nicht gelten, k\"onnte es zu Problemen bei modernen kryptographischen Verfahren kommen, da diese auf immer gr\"o\ss er werdenden Primzahlen beruhen. 
		
		\item  Ebenfalls von Euklid kommt der Beweis daf\"ur, dass sich jede nat\"urliche Zahl als Produkt von Primzahlen darstellen l\"asst. Sp\"ater entdeckte Gauss, dass diese sogenannte Zerlegung bis auf die Reihenfolge der einzelnen Elemente eindeutig ist. \\
		Dies ist besonders interessant, weil es uns erm\"oglicht, zusammengesetzte Zahlen (also Zahlen, die nicht prim sind) daran zu erkennen, dass wir einen Faktor ungleich der Zahl oder 1 gefunden haben. Au\ss erdem birgt dies die Grundlage zu manchen Angriffen auf kryptografische Verfahren wie RSA, auf die in diesem Text allerdings nicht weiter eingegangen wird.
	\end{itemize}
	
	\noindent Dieser Text wird sich im weiteren Verlauf vor allem mit der Zerlegung von Zahlen in deren Primfaktoren befassen. 